\documentclass[../doc.tex]{subfiles}
\begin{document}
\asection{5}{Функциональные требования}
\par
Cистема должна позволять Пользователю:
\begin{itemize}
    \item Создать учетную запись, используя пару email-пароль.
    \item Войти в систему, использую пару email-пароль.
    \item Редактировать личные данные, включая email и пароль.
    \item На странице "Профиль" включать/отключать уведомления о новых поступлениях.
    \item На странице "Профиль" включать/отключать уведомления об истечении срока бронирования.
    \item Получать уведомления на email о новых поступлениях.
    \item Получать уведомления на email об истечении срока бронирования.
    \item Получать уведомления на email о книгах, на получение которых Пользователь ранее вставал в очередь.
    \item Получать актуальный список имеющейся литературы.
    \item Формировать поисковые запросы по названию, автору, ключевым словам по базе библиотеки.
    \item Просматривать страницу книги с информацией о ней.
    \item Брать книгу с добавлением соответствующей записи в читательскую карточку.
    \item Возвращать книгу с добавлением соответствующей записи в читательскую карточку.
    \item Видеть, у кого находится книга, если она занята.
    \item Записываться в очередь на книгу.
    \item Скачивать электронное издание на странице соответствующей книги (при его наличии).
    \item Оставлять отзыв на странице книги.
    \item Удалять оставленные отзывы.
    \item Просматривать отзывы других пользователей.
    \item Добавлять книгу в список прочитанных в качестве "ранее прочитанной".
    \item Получать список книг с датой истечения срока бронирования, находящихся у Пользователя на руках.
    \item Получать список книг, прочитанных Пользователем. 
    \item Формировать запросы на добавление книги в библиотеку.
    \item Голосовать в запросах на книги других пользователей.
    \item Менять цветовую схему интерфейса.
    \item Выйти из системы.
\end{itemize}
\par
Система должна позволять Администратору:
\begin{itemize}
    \item Совершать все действия, доступные Пользователю.
    \item Добавлять/редактировать/удалять информацию о книгах вручную.
    \item Добавлять книгу, используя ISBN.
    \item Добавлять книги, используя Excel/CSV файлы.
    \item Загружать электронные издания.
    \item Удалять отзывы Пользователей.
    \item Принимать запросы на добавление книг.
    \item Устанавливать ограничение на количество и срок бронирований.
    \item Получать данные о Пользователях, просрочивших дату возврата книги.
\end{itemize}
\end{document}