\documentclass[./tools.tex]{subfiles}

\begin{document}
  \subsubsection{Frontend}
  \begin{enumerate}
    \item Typescript
    \par
    Язык программирования на основе JavaScript. Развивается компанией Microsoft как open-source продукт. Благодаря большому сообществу существует огромное количество библиотек, ускоряющих разработку приложений.
    \par
    Строгая типизация обеспечивает консистентность данных, хранимых клиентским приложением. Реализует многие концепции ООП. Упрощает масштабирование и поддержку клиентских приложений.
    \item Vue 3
    \par
    Open-source фреймворк для разработки SPA приложений. Написан на языке TypeScript.
    \par
    Преимущества Vue 3:
    \begin{itemize}
      \item Легковесность
      \item Гибкость
      \item Скорость и производительность.
      \item Использование компонентного подхода с использованием HTML - шаблонов.
    \end{itemize}
    \item InertiaJS
    \par
    Библиотека для взаимодействия с backend. Позволяет организовать маршрутизацию, работу с данными и состоянием приложения на стороне сервера. Ускоряет разработку SPA на Vue, так как не требуется организовывать работу с данными на клиентской стороне с помощью хранилищ глобального состояния как Pinia или Vuex.
    \item ZiggyJS 
    \par
    Библиотека для использования именованных Laravel-маршрутов. При использовании InertiaJS обеспечивает согласованность именования маршрутов на клиенте и сервере.
    \item TailwindCSS
    \par
    Tailwind CSS — это CSS-фреймворк, предлагающий обширный каталог классов и инструментов для облегчения стилизации сайта или приложения. Tailwind предлагает предварительно разработанные виджеты для создания веб-интерфейса, а не предварительно стилизованные адаптивные компоненты, что делает разработку интерфейса более гибкой.
    \item DaisyUI
    \par
    Плагин для TailwindCSS. Представляет из себя дизайн-систему, организованную в виде CSS-классов с использованием TailwindCSS классов и утилит.
  \end{enumerate}
\end{document}