\documentclass[./project.tex]{subfiles}

\begin{document}
\asubsection{3}{Модули и алгоритмы}
\subsubsection{Backend}
\par
Серверная часть сервиса состоит из 12 модулей:
\begin{itemize}
  \item Controllers – имеет три подмодуля, состоящих из наборов классов - контроллеров для обработки пользовательских запросов
  \begin{itemize}
    \item Auth – функционал, связанный с авторизацией и аутентификацией.
    \item Books – функционал, связанный с доменной областью сервиса.
    \item Admin – функционал, доступный только для Администраторов. 
  \end{itemize}
  \item Console – набор классов для создания пользовательских команд (отправка уведомлений).
  \item Requests – схемы для валидации пользовательских запросов.
  \item Mail – набор классов для реализации функционала по отправке электронных писем.
  \item Models – набор классов для взаимодействия с базой данных с помощью EloquentORM.
  \item Policies – набор классов для разграничения прав доступа для различных сущностей базы данны.
  \item Rules – кастомные правила валидации.
  \item Services – набор классов, реализающих бизнес–логику сервиса. Каждой крупной модели соответсвует свой класс–сервис.
  \item config – модуль, содержащий конфигурационные файлы.
  \item database – имеет три подмодуля:
  \begin{itemize}
    \item factories – набор классов, определяющих фабрики для классов–моделей. Экземпляров классов заполняются случайными данными, описанными в схеме фабрики.
    \item migrations – содержит файлы-миграции базы данных.
    \item seeders – набор классов для наполнения базы данных тестовыми данными. Данный подмодуль использует подмодуль factories.
  \end{itemize}
  \item lang – содержит файлы для локализации сервиса.
  \item routes – каталог, содержащий web и api-маршруты.
\end{itemize}
\subsubsection{Frontend}
\par
Клиентская часть сервиса состоит из 6 модулей:
\begin{itemize}
  \item components – включает 7 подмодулей:
  \begin{itemize}
    \item Admin – компоненты для интерфейсов администратора.
    \item Auth – формы аутентификации, регистрации.
    \item Base – переиспользуемые внутри проекта UI компоненты.
    \item Books – компоненты, связанные с ключевой доменной областью сервиса.
    \item Profile – компоненты, используемые на странице «Профиль».
    \item Svg – иконки.
  \end{itemize}
  \item layouts – компоненты, используемые для создания базовой структуры страницы.
  \item pages – страницы сервиса.
  \item types – объявления типов, интерфейсов, связанных с моделями данных на серверной части сервиса.
  \item utils – функции-утилиты, используемые внутри компонентов.
\end{itemize}
\end{document}