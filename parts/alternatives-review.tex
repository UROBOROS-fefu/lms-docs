\documentclass[./intro.tex]{subfiles}
\usepackage{array}
\begin{document}
\subsection{Обзор существующих методов решения}
\parДля анализа выбраны следующие программные продукты:
\begin{enumerate}
    \item CodeAchi \cite{Codeachi} \\
    Одна из самых востребованных платформ на зарубежном рынке. Представляет собой оконное приложение (Windows). Благодаря обширному функционалу и интуитивному интерфейсу популярна среди публичных библиотек, школ, колледжей и университетов.
    \item Alexandria \cite{Alexandria} \\
    Веб-сайт, ориентированный на людей любых возрастов. Подходит как для небольших, так и для крупных библиотек. В дополнение, существует детский раздел.
    \item Mandarin M5 \cite{Mandarin} \\
    Представляет собой веб-интерфейс. Дополнительные модули позволяют библиотекам настраивать свою систему в соответствии со своими индивидуальными потребностями. Также платформа поддерживает Unicode, что даёт возможность пользователям искать материалы на разных языках, включая китайский, арабский и т. д.
    \item Handy Library Manager \cite{Handy} \\
    Гибридное локальное и облачное решение системы управления библиотекой с доплатой за последнее. В отличие от других платформ, поддерживает хранение неограниченного количества книг и их катологизацию. 
    \item KnowAll Matrix \cite{KnowAll} \\
    Представляет собой веб-интерфейс. Поддерживает хранение как печатных, так и электронных изданий.
    \item 1С: Библиотека \cite{1C} \\
    Продукт позволяет автоматизировать рабочие процессы библиотеки в зависимости от ее назначения, типа, состава фондов. Имеется виртуальный кабинет читателя. Также может быть интегрирован с другими типовыми решениями фирмы "1С".
    \item OPAC-Global \cite{OPAC} \\
    Информационная система, основанная на облачных технологиях. Её можно развернуть как для отдельной небольшой библиотеки, так и для сети библиотек. Систему используют в Президентской библиотеке имени Б. Н. Ельцина, научной библиотеке Государственной Третьяковской галереи, Центральной научной библиотеке ПМГМУ им. И. М. Сеченова и других.
    \item Либэр \cite{Liber} \\
    Имеется два варианта исполнения: оконное приложение (Windows, MacOS, Linux) и браузерное решение. Реализована возможность работы сети библиотек с разными базами или единой базой.
    \item АБИС "Марк" \cite{ABISmark} \\
    Представляет собой оконное приложение (Windows, MacOS, Linux). У программы имеется облачная версия. Поддерживает различные популярные СУБД, можно гибко настроить и расширить функциональные возможности благодаря использованию встроенных скриптов.
    \item Проектируемое решение
\end{enumerate}

\begin{table}[h]
    \caption{Анализ программных продуктов-аналогов}
    \label{table:1}
    \begin{tabular}{|p{8.5cm}|c|c|c|c|c|c|c|c|c|c|}
        \hline
        Критерии & 1 & 2 & 3 & 4 & 5 & 6 & 7 & 8 & 9 & 10\\
        \hline
        Автоматизированный процесс получения и возврата изданий & – & + & – & – & – & – & – & – & – & + \\
        \hline
        Возможность формировать запросы на добавление книг & – & – & – & – & + & – & – & + & + & + \\
        \hline
        Заполнение базы данных различными способами (ISBN, CSV, XLS) & + & – & – & + & + & + & + & + & + & + \\
        \hline
        Отслеживание задолженностей & + & + & + & - & + & + & + & + & + & + \\
        \hline
        Доступ к электронным изданиям & – & – & – & + & + & + & + & + & + & + \\
        \hline
        Возможность оставлять рецензии на книги & – & + & + & – & + & – & – & – & – & + \\
        \hline
        Формирование очередей на получение книги & – & – & – & – & – & + & – & + & – & + \\
        \hline
        Уведомления о новых поступлениях & – & – & – & + & + & – & – & – & – & + \\
        \hline
        Уведомления об истечении срока брони & + & + & + & + & + & – & + & – & + & + \\
        \hline
        Наличие русской локализации & – & – & – & – & – & + & + & + & + & + \\
        \hline
    \end{tabular}    
\end{table}

\par На основании проведённого анализа перечисленных решений сделан вывод. 5 из 9 решений не имеют русской локализации, так как ориентированы на зарубежный рынок.
Продукты, разработанные российскими компаниями, зачастую не соответствуют необходимым функциональным требованиям и давно не обновляются, что затрудняет их использование на современных устройствах.
В связи с этим, есть необходимость в разработке продукта, ориентированного на российский рынок и современные требования.
\end{document}