\documentclass[./intro.tex]{subfiles}
\usepackage{array}
\begin{document}
\subsection{Обзор существующих методов решения}
\parДля анализа выбраны следующие программные продукты:
\begin{enumerate}
    \item CodeAchi
    \item Alexandria
    \item Mandarin M5
    \item Handy Library Manager
    \item KnowAll Matrix
    \item 1С:Библиотека
    \item OPAC-Global
    \item Либэр
    \item АБИС "Марк"
    \item Проектируемое решение
\end{enumerate}

\begin{table}[h]
    \caption{Анализ программных продуктов-аналогов}
    \label{table:1}
    \begin{tabular}{|p{8.5cm}|c|c|c|c|c|c|c|c|c|c|}
        \hline
        & 1 & 2 & 3 & 4 & 5 & 6 & 7 & 8 & 9 & 10 \\
        \hline
        Автоматизированный процесс получения и возврата изданий & - & + & - & - & - & - & - & - & - & + \\
        \hline
        Возможность формировать запросы на добавление книг & - & + & - & - & + & - & - & + & + & + \\
        \hline
        Заполнение базы данных различными способами (ISBN, CSV, XLS) & + & - & - & + & + & + & + & + & + & + \\
        \hline
        Отслеживание задолженностей & + & + & + & - & + & + & + & + & + & + \\
        \hline
        Доступ к электронным изданиям & - & + & - & + & + & + & + & + & + & + \\
        \hline
        Возможность оставлять рецензии на книги & - & - & + & - & + & - & - & - & - & + \\
        \hline
        Формирование очередей на получение книги & - & - & - & - & - & + & - & + & - & + \\
        \hline
        Регулярная отправка уведомлений на эл. почту & + & + & + & + & + & - & + & - & + & + \\
        \hline
        Интуитивно-понятный интерфейс & + & + & - & - & + & - & - & - & + & + \\
        \hline
        Наличие русской локализации & - & - & - & - & + & + & + & + & + & + \\
        \hline
    \end{tabular}    
\end{table}


\par На основании проведенного анализа перечисленных решений сделан вывод. 4 из 9 решений не имеют русской локализации, так как ориентированы на зарубежный рынок.
Продукты, разработанные российскими компаниями, зачастую не соответствуют необходимым функциональным требованиям и давно не обновляются, что затрудняет их использование на современных устройствах.
В связи с этим, есть необходимость в разработке продукта, ориентированного на российский рынок и современные требования.
\end{document}