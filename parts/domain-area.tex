\documentclass[./intro.tex]{subfiles}
\begin{document}
\subsection{Описание предметной области}
\par 
В современных компаниях все чаще появляются корпоративные библиотеки с профильной литературой.
Объем изданий в таких библиотеках может достигать нескольких сотен, вследствие чего возникает необходимость в ведении учета литературы.
\par
Применяемые подходы к решению задачи по учету ресурсов корпоративной библиотеки:
\begin{itemize}
    \item Ведение учета на бумажных носителях
    \item Сервисы онлайн-форм (пр. Google Forms)
    \item Ведение таблиц (пр. Excel, Google Sheets)
\end{itemize}
\par Описанные методы решения поставленной задачи ограничивают использование ресурсов корпоративной библиотеки и не обеспечивают должного учета литературных изданий, поэтому требуется применение специализированных программных решений.
\par 
Существующие информационные системы для библиотек в большинстве случаев не применимы к корпоративным библиотекам.
Значительное число решений требуют наличие библиотекаря, обеспечивающего выдачу, поддержку актуальности списка имеющихся изданий и т. д.
\par
В связи с этим появляется необходимость в проектировании и разработке системы, автоматизирующей процессы взаимодействия с имеющимися ресурсами литературы и не требующей наличия библиотекаря. Полностью исключить
\par
\end{document}