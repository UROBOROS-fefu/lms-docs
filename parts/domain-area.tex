\documentclass[./intro.tex]{subfiles}
\begin{document}
\subsection{Описание предметной области}
\par 
В современных компаниях все чаще появляются корпоративные библиотеки с профильной литературой.
Объем изданий в таких библиотеках может достигать нескольких сотен, вследствие чего возникает необходимость в ведении учета литературы.
\par
Применяемые подходы к решению задачи по учету ресурсов корпоративной библиотеки:
\begin{itemize}
    \item Ведение учета на бумажных носителях
    \item Сервисы онлайн-форм (пр. Google Forms)
    \item Ведение таблиц (пр. Excel, Google Sheets)
\end{itemize}
\par Описанные методы решения поставленной задачи ограничивают использование ресурсов корпоративной библиотеки и не обеспечивают должного учета литературных изданий, поэтому требуется применение специализированных программных решений.
\par При ведении учета на бумажных носителях не обеспечивается требуемая целостность данных.
Также этот подход требует наличия библиотекаря, организующего рабочий процесс. 
В большинстве случаев должность библиотекаря не предусмотрена в предприятии, поэтому данный подход используется в редких случаях.
\par Сервисы онлайн-форм: в сравнении с бумажными носителями сохраняется целостность данных, однако нет централизованного хранения и структуризации информации об изданиях и читателях.
Данный подход позволяет вести только учет изданий, находящихся на руках.
\par Ведение электронных таблиц: таблицы позволяют организовать базу данных корпоративной библиотеки и хранить большинство требуемой информации. 
Однако недостатком этого подхода является закрытость, так как версия с актуальными данными хранится локально.
При общем доступе для сохранения целостности данных требуется разграничение прав доступа, что не всегда реализуемо удобным способом.
\par 
Существующие информационные системы для библиотек в большинстве случаев не применимы к корпоративным библиотекам.
Значительное число решений требуют наличие библиотекаря, обеспечивающего выдачу, поддержку актуальности списка имеющихся изданий и т. д.
\par
В связи с этим появляется необходимость в проектировании и разработке системы, автоматизирующей процессы взаимодействия с имеющимися ресурсами литературы и не требующей наличия библиотекаря.
\par
Ключевые преимущества проектируемой системы:
\begin{itemize}
    \item Автоматизированный процес выдачи и возврата книг - устраняет необходимость в библиотекаре
    \item Общедоступный систематизированный электронный каталог - ускоряет процесс поиска необходимого издания
    \item Централизованный доступ к электронным изданиям
\end{itemize}
\end{document}